%%%%%%%%%%%%%%%%%%%%%%%%%%%%%%%%%%%%%%%%%%%%%%%%%%%%%%%%%%%%%%%%%%%%
\chapter{Introduction}
\label{chap:1}
%%%%%%%%%%%%%%%%%%%%%%%%%%%%%%%%%%%%%%%%%%%%%%%%%%%%%%%%%%%%%%%%%%%%
%
As excitons have higher photo-emission rates than free carriers in semiconductors,
highly efficient light-emitting devices utilizing excitons are expected.
%
Our exciton dynamics simulation model based on phononic,
electronic, and radiative processes,
named the phononic--excitonic--radiative (PXR) model,
is available to analyze the dynamics under both thermal equilibrium and nonequilibrium states
\cite{CK-JL-2022, CK-MSSP-2022}%
.
%

%
According to Ref. \cite{CK-JL-2022},
the population transitions to upper levels of excitons reduce
the radiative recombination rate of excitons in a three-dimensional (3D) space.
%
In bulk crystals,
the rates of the radiative deexcitation and exciton-formation transitions
from the continuum are lower than 0.01\% of the phononic transition rates.
%





%%%%%%%%%%%%%%%%%%%%%%%%%%%%%%%%%%%%%%%%%%%%%%%%%%%%%%%%%%%%%%%%%%%%
\chapter{Theoretical model}
\label{chap:2}
%%%%%%%%%%%%%%%%%%%%%%%%%%%%%%%%%%%%%%%%%%%%%%%%%%%%%%%%%%%%%%%%%%%%
%

%
\section{Energy eigenstates}
%

%
% Table 1
\begin{table}
    \centering
    \caption{%
        Mott densities of 2D excitons with $n_{\parallel}$ estimated by
        $N_{\mathrm{M}}(n_{\parallel}) = \frac{1}{4} \frac{1}{ [ (2n_{\parallel} - 1) a_{X\parallel}^{\mathrm{2D}} ]^2}$.
    }
    \label{tab:1}
    \begin{tabular}[htbp]{@{}ccc@{}}
        \hline
        $n_{\parallel}$
        & $N_{\mathrm{M}}(n_{\parallel})$ of $E_{\mathrm{B}}(1) = 108$ meV
        & $N_{\mathrm{M}}(n_{\parallel})$ of $E_{\mathrm{B}}(1) = 215$ meV \\
        %
        \hline
        %
        1   & $1.276 \times 10^{13}$ (cm$^{-2}$)    & $3.863 \times 10^{13}$ (cm$^{-2}$) \\
        2	& $1.418 \times 10^{12}$ (cm$^{-2}$)    & $4.292 \times 10^{12}$ (cm$^{-2}$) \\
        3	& $5.106 \times 10^{11}$ (cm$^{-2}$)    & $1.545 \times 10^{12}$ (cm$^{-2}$) \\
        4	& $2.605 \times 10^{11}$ (cm$^{-2}$)    & $7.883 \times 10^{11}$ (cm$^{-2}$) \\
        %
        \hline
    \end{tabular}
\end{table}
%

%
Table \ref{tab:1} shows Mott densities of excitons with $n_{\parallel}$
under two conditions of our simulation.
%
The wavefunction of an exciton state is denoted as
\begin{equation}
    \Psi_{X}(z_e, z_h, \bm{r}_{\parallel}, \bm{R}_{\parallel})
    = \phi_{\perp}(z_e, z_h)
        \Psi_{X\parallel}^{\nu_{\parallel},\bm{K}_{X\parallel}}
        (\bm{r}_{\parallel}, \bm{R}_{\parallel} ),
\end{equation}
where
$\Psi_{X\parallel}^{\nu_{\parallel},\bm{K}_{X\parallel}}
    (\bm{r}_{\parallel}, \bm{R}_{\parallel} )$
is the 2D-exciton wavefunction confined in a QW;
$\bm{R}_\parallel$ and $\bm{r}_\parallel$ are the coordinate vectors
of the CM and relative motions of the electron and hole of a 2D exciton, respectively;
$\bm{K}_{X\parallel}$ is the wavevector of the 2D exciton;
and $\nu_{\parallel}$ represents a set of quantum numbers specifying an excitonic quantum state.
%
The wavefunction
$\Psi_{X\parallel}^{\nu_{\parallel},\bm{K}_{X\parallel}}
    (\bm{r}_{\parallel}, \bm{R}_{\parallel} )$
is obtained from the Schr\"{o}dinger equation of 2D excitons, which is given by
%
\begin{equation}
    \left[
        - \frac{\hbar^2}{2M_{\parallel}}  \nabla_{\bm{R}_{\parallel}}^2
        - \frac{\hbar^2}{2\mu_{\parallel}}\nabla_{\bm{r}_{\parallel}}^2
        - \frac{ e_0^2 }{ 4\pi\varepsilon_{\mathrm{s}} r_{\parallel} }
    \right]
    \Psi_{X\parallel}^{\nu_{\parallel},\bm{K}_{X\parallel}}
    (\bm{r}_{\parallel}, \bm{R}_{\parallel} )
    =
    E_X^{\mathrm{tot}}(K_{X\parallel}, n_{\parallel})
    \Psi_{X\parallel}^{\nu_{\parallel},\bm{K}_{X\parallel}}
    (\bm{r}_{\parallel}, \bm{R}_{\parallel} ).
\end{equation}
%
The respective $\nabla_{\bm{R}_{\parallel}}$ and $\nabla_{\bm{r}_{\parallel}}$ are
the nabla differential operators with respect to the components of the vectors
$\bm{R}_{\parallel}$ and $\bm{r}_{\parallel}$.
%
$e_0$ is the elementary charge.
%
The effective dielectric constant $\varepsilon_{\mathrm{s}}$ is
$ e_0^2 / [ 4 \pi a_{X\parallel}^{\mathrm{2D}} E_{\mathrm{B}}(1) ]$,
where respective $a_{X\parallel}^{\mathrm{2D}}$ and $E_{\mathrm{B}}(1)$
are the 2D-$1S$-exciton Bohr radius and binding energy.
%
Here, Equation \eqref{eq:5} can be analytically solved
by the method of separation of variables of $\bm{R}_{\parallel}$ and $\bm{r}_{\parallel}$ as follows:
%
\begin{equation}
    \Psi_{X\parallel}^{\nu_{\parallel},\bm{K}_{X\parallel}}
    (\bm{r}_{\parallel}, \bm{R}_{\parallel} )
        = \phi_{S_X}^{\bm{K}_{X\parallel}}(\bm{R}_{\parallel})
          \psi_{\nu_{\parallel}}(\bm{r}_{\parallel}),
\end{equation}
%
where $S_X$ is the normalization area of excitons.
%
$\phi_{S_X}^{\bm{K}_{X\parallel}}(\bm{R}_{\parallel})$ is the 2D-plane wave defined as Equation \eqref{eq:2}
of $\bm{k}_{\parallel} = \bm{K}_{X\parallel},\ \bm{r}_{\parallel} = \bm{R}_{\parallel}$, and $S = S_X$.
%
$\psi_{\nu_{\parallel}}(\bm{r}_{\parallel})$ is the 2D-hydrogen-atom wavefunction
determined by a set of two quantum numbers $\nu_{\parallel} = (n_{\parallel}, m_{\parallel})$,
\textsuperscript{\cite{35-Shinada-JPSJ-1966,36-Yang-PRA-1991,37-Parfitt-JMP-2002}}
where respective $n_{\parallel}$ $(= 1, 2, \dots)$ and $m_{\parallel}$ are
the principal quantum number and the azimuthal quantum number,
which are integers satisfying $|m_{\parallel}| < n_{\parallel}$.
%

%
The total energy of the exciton state with $\bm{K}_{X\parallel}$ and $n_{\parallel}$ is given by
%
\begin{equation}
    E_X^{\mathrm{tot}}(K_{X\parallel}, n_{\parallel})
    = E_{X\parallel}(K_{X\parallel}) - E_{\mathrm{B}}(n_{\parallel}),
\end{equation}
%
where $E_{X\parallel}(K_{X\parallel}) = \hbar^2 K_{X\parallel}^2 / 2M_{\parallel}$ is the kinetic energy of 2D excitons.
%
$E_{\mathrm{B}}(n_{\parallel}) = E_{\mathrm{B}}(1) / ( 2 n_{\parallel} - 1 )^2$
is the binding energy of a 2D exciton with $n_{\parallel}$.
%
The 2D-$1S$-exciton binding energy $E_{\mathrm{B}}(1)$ is
four times larger than the 3D-$1S$-exciton binding energy $E_{\mathrm{B}\parallel}^{\mathrm{3D}}$.
%
Figure \ref{fig:1} shows the energy levels of 2D excitons and continuum states.
%



%
% figure 1
\begin{figure}[t]
    \includegraphics[width=\linewidth]{fig/Fig1.pdf}
    \caption{
        This is the caption of figure 1.
    }
    \label{fig:1}
\end{figure}
%

